\documentclass[../main.tex]{subfiles} 
\usepackage{ctex}
\usepackage{xltxtra}
\usepackage{graphicx}
\usepackage{booktabs}
\usepackage{amsmath}
\usepackage{mathdots}
\usepackage{amssymb}
\usepackage{cite}
\usepackage{appendix}
\usepackage{array}
\usepackage{subfigure}
\begin{document}

    \subsection{分类算法}

        \subsubsection{SVC}

            SVM即Support Vector Machine,而SVC也就是用于分类的支持向量机模型。

            首先在样本空间中,由于通常面对的就是一个二分类问题,因此需要寻找一个用于划分的鲁棒性最强的超平面,其可以被定义为:

            \begin{equation*}
                w^Tx+b = 0
            \end{equation*}

            由点到超平面的距离公式,可以得到优化目标:

            \begin{equation*}
                \min \limits_{w,b} \quad \frac{1}{2} \|w\|^2
            \end{equation*}

            而通常样本数据是交杂在一起的,此时该分类问题需要使用软间隔的SVM。软间隔的基本思想就是:允许模型犯错误,并不要求完全的线性可分,样本只在一定范围外记入错误,其他犯错的样本视而不见.

            可以推导出SVM模型的一般形式如下:

            \begin{equation*}
                \begin{aligned}
                &\min \limits_{f} \Omega(f) + c \sum \limits_{i=1}^m l(f(x_i),y_i)\\
                &\text{s.t.} \quad y_i(w^Tx_i + b) \geqslant 1 , i = 1,2,\cdots,n
                \end{aligned}
            \end{equation*}

            其中$\Omega$指代结构风险,例如间隔距离,c指的是惩罚因子,$l$指的是经验风险,用来描述模型与训练数据的契合程度,可以通过调整惩罚因子c来描述允许模型犯错误的程度,c越小就是允许犯更多的错误.

            同时SVM的训练过程中可以使用核函数进行非线性分类,在这里使用的是高斯核。因为高斯核可以把数据拓展到无限维,在无限维的时候数据是更有可能可分的,所以认为高斯核在该问题上的训练效果会比线性核训练的 SVM 效果好一点.

        \subsubsection{Decision Tree}

            相比起其他模型,决策树应该是一种较为具体的方法,在训练过程中根据信息增益、增益率或是基尼指数等其他指标为标准进行数据的划分,生成一种树形结构,其中每个内部节点表示一个属性上的判断,每个分支代表一个判断结果的输出,最后每个叶节点代表一种分类结果。
            同时决策树还是后续集成学习AdaBoost,CatBoost,Random Forest等模型的基础。
        
        \subsubsection{Naive Bayes}

            朴素贝叶斯和其他绝大多数的分类算法都不同。对于大多数的分类算法,比如决策树,KNN,逻辑回归,支持向量机等,他们都是判别方法,也就是直接学习出特征输出Y和特征X之间的关系。但是朴素贝叶斯却是生成方法,也就是直接找出特征输出Y和特征X的联合分布。
            准确来说,根据已有样本进行贝叶斯估计学习出先验概率P(Y)和条件概率P(X|Y),进而求出联合分布概率P(XY),最后利用贝叶斯定理求解P(Y|X),也就是说,它尝试去找到底这个数据是怎么生成的,然后再进行分类。哪个类别最有可能产生这个标签特征,该数据就属于那个类别。

    \subsection{聚类算法}
    
        \subsubsection{KNN}

            KNN,即K Nearest Neighbors。KNN的原理非常简单,就是对于每个新的需要预测的样本查看离它最近的k个样本的类别来判断该样本是什么类别。距离度量通常使用欧氏距离。

    \subsection{集成学习}

        集成方式主要分为两大类:并行和串行。其中并行主要有使用随机子样本的Bagging和随机子空间的随机森林,串行主要使用Boosting,代表算法有AdaBoost等。

        \subsubsection{Random Forest}

            随机森林的子模型是决策树。
        
            \textcircled{1}在每一次训练的过程中有放回地随机选择N个样本训练一个决策树.
        
            \textcircled{2}随机选取m个属性,根据属性选取的策略(如信息增益、基尼指数)选择一个属性,对于决策树的结点进行分裂.
    
            \textcircled{3}重复步骤2直到不能分裂.
    
            \textcircled{4}重复步骤1、2、3建立大量决策树,构建随机森林.

        \subsubsection{CatBoost}

            CatBoost的基础是AdaBoost,而AdaBoost首先先将分布的权重设为相等的$\frac{1}{m}$,然后根据设计的训练轮数T,在每一轮的训练中,根据之前的分布从D中训练出基模型,计算出加权错误率,并依据加权错误率计算出样本的权重,错误的样本的权重更大,并依据样本的权重更新分布,进入下一轮训练。
        
            CatBoost在建立树形结构的过程中使用的是构建对称(平衡)树。在每一步中,前一棵树的叶子都使用相同的条件进行拆分,并且选择损失最低的特征分割对并将其用于所有级别的节点,因此CatBoost有正则作用且预测极快。

            除此以外,CatBoost提供更强大的对类别特征的支持,直接支持字符串类型的类别特征,无需预处理,但是在前文中已经提前进行了类别编码,因此后续重新尝试了不提前进行类别编码直接进行训练,并观察对比两者的训练效果。

\end{document}